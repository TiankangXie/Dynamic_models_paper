%%%%%%%%%%%%%%%%%%%%%%%%%%%%%%%%%%%%%%%%%
% Journal Article
% LaTeX Template
% Version 1.3 (9/9/13)
%
% This template has been downloaded from:
% http://www.LaTeXTemplates.com
%
% Original author:
% Frits Wenneker (http://www.howtotex.com)
%
% License:
% CC BY-NC-SA 3.0 (http://creativecommons.org/licenses/by-nc-sa/3.0/)
%
%%%%%%%%%%%%%%%%%%%%%%%%%%%%%%%%%%%%%%%%%
%----------------------------------------------------------------------------------------
%       PACKAGES AND OTHER DOCUMENT CONFIGURATIONS
%----------------------------------------------------------------------------------------
\documentclass[paper=letter, fontsize=12pt]{article}
\usepackage[english]{babel} % English language/hyphenation
\usepackage{amsmath,amsfonts,amsthm} % Math packages
\usepackage[utf8]{inputenc}
\usepackage{float}
\usepackage{lipsum} % Package to generate dummy text throughout this template
\usepackage{blindtext}
\usepackage{graphicx} 
\usepackage{caption}
\usepackage{subcaption}
\usepackage[sc]{mathpazo} % Use the Palatino font
\usepackage[T1]{fontenc} % Use 8-bit encoding that has 256 glyphs
\linespread{1.05} % Line spacing - Palatino needs more space between lines
\usepackage{microtype} % Slightly tweak font spacing for aesthetics
\usepackage[hmarginratio=1:1,top=32mm,columnsep=20pt]{geometry} % Document margins
\usepackage{multicol} % Used for the two-column layout of the document
%\usepackage[hang, small,labelfont=bf,up,textfont=it,up]{caption} % Custom captions under/above floats in tables or figures
\usepackage{booktabs} % Horizontal rules in tables
\usepackage{float} % Required for tables and figures in the multi-column environment - they need to be placed in specific locations with the [H] (e.g. \begin{table}[H])
\usepackage{hyperref} % For hyperlinks in the PDF
\usepackage{lettrine} % The lettrine is the first enlarged letter at the beginning of the text
\usepackage{paralist} % Used for the compactitem environment which makes bullet points with less space between them
\usepackage{abstract} % Allows abstract customization
\renewcommand{\abstractnamefont}{\normalfont\bfseries} % Set the "Abstract" text to bold
\renewcommand{\abstracttextfont}{\normalfont\small\itshape} % Set the abstract itself to small italic text
\usepackage{titlesec} % Allows customization of titles

\renewcommand\thesection{\Roman{section}} % Roman numerals for the sections
\renewcommand\thesubsection{\Roman{subsection}} % Roman numerals for subsections

\titleformat{\section}[block]{\large\scshape\centering}{\thesection.}{1em}{} % Change the look of the section titles
\titleformat{\subsection}[block]{\large}{\thesubsection.}{1em}{} % Change the look of the section titles
\newcommand{\horrule}[1]{\rule{\linewidth}{#1}} % Create horizontal rule command with 1 argument of height
\usepackage{fancyhdr} % Headers and footers
\pagestyle{fancy} % All pages have headers and footers
\fancyhead{} % Blank out the default header
\fancyfoot{} % Blank out the default footer

\fancyhead[C]{F003KCV \tab \tab } % Custom header text

\fancyfoot[RO,LE]{\thepage} % Custom footer text
%----------------------------------------------------------------------------------------
%       TITLE SECTION
%----------------------------------------------------------------------------------------
\title{\vspace{-15mm}\fontsize{24pt}{10pt}\selectfont\textbf{QBS 130 HW2 }} % Article title

\date{}

%----------------------------------------------------------------------------------------
\begin{document}
\maketitle % Insert title
\thispagestyle{fancy} % All pages have headers and footers


\section{P1}

a. Mortality Rate
\newline
b. Period Prevalence
\newline
c. Cumulative Incidence
\newline
d. Case Fatality Rate
\newline
e. Cumulative Incidence 
\newline
f. Incidence Rate
\newline
g. Period Prevalence 
\newline
h. Point Prevalence
\newline
i. Cumulative Incidence

\newpage

\section{P2}
\subsection{a}
%To be continued

\subsection{b}
$Case Fatality = \frac{6080}{62840} = 0.0968$
The case fatality rate is 968 per 10,000 cases.

\subsection{c}
$Case Fatality_{whites} = \frac{2940}{17850} = 0.165$
\newline
$Case Fatality_{blacks} = \frac{410}{2360} = 0.174$
\newline
The case Fatality rate among adults \geq 65 is 165 per 1000 cases for white people and 174 per 1000 cases for blacks.

\newpage
\section{P3}
a. True
\newline
b. False
\newline
c. False
\newline
d. False
\newline
e. False

\newpage
\section{P4}
\subsection{a}
Case control study. Because we identify cases where disease has already developed and we look back to history.
\subsection{b}
Retrospective cohort study. Because we want to measure exposure and we do this by examining records
\subsection{c}


\newpage
\section{P5}
\subsection{a}
The purpose of randomization is to try to remove bias in an experiment
\subsection{b}

\subsection{c}
It means that prior to the study the researcher is uncertain about whether a treatment will be beneficial or not. 
It is important because otherwise the researcher will be biased towards one treatment. If some accidental evidence ,which benefits the preferred treatment, is gathered, the researcher will immediately choose the preferred treatment.
%----------------------------------------------------------------------------------------
%\end{multicols}
\end{document}
